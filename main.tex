\documentclass[13pt, usenames,dvipsnames]{beamer} %
% \renewcommand{\baselinestretch}{1.3}
\usepackage[T1]{fontenc}
\usepackage{lmodern}
\usepackage{minted}
\usepackage{xcolor}
\usepackage{etoolbox}
\usepackage{tikz}
\usetikzlibrary{fit}
\usepackage{color, colortbl}
\usepackage{soul}
\usepackage{fontspec}
\usepackage{fontawesome}
\usepackage{array}
\usepackage{hyperref}

% \usepackage[export]{adjustbox}

\usetikzlibrary{mindmap,trees,shadows,shapes,positioning,arrows,calc,matrix}
\tikzset{
  invisible/.style={opacity=0},
  visible on/.style={alt={#1{}{invisible}}},
  alt/.code args={<#1>#2#3}{%
    \alt<#1>{\pgfkeysalso{#2}}{\pgfkeysalso{#3}} % \pgfkeysalso doesn't change the path
  },
}
\setmonofont[
  Contextuals={Alternate}
  ]{FiraCode-Regular}




\definecolor{aliceblue}{rgb}{0.94, 0.97, 1.0}
\definecolor{beige}{rgb}{0.96, 0.96, 0.86}
\definecolor{chartreuse}{rgb}{0.87, 1.0, 0.0}

\sethlcolor{yellow}

\usemintedstyle{manni}
\setminted{fontsize=\footnotesize, baselinestretch=1}

\setbeamertemplate{footline}[frame number]
\setbeamercolor{normal text}{fg=black!80,bg=white}

\newcommand{\mycode}[2][\tiny] {\mintinline[bgcolor=gray!10, fontsize=#1]{python}{#2}}

\title{Parallel Computing in Python}
\subtitle{Current State and Recent Advances}
\author{\href{http://kbroman.org}{Pierre Glaser}
\href{https://github.com/pierreglaser}{\faGithub}
\href{https://twitter.com/pierreglaser}{\faTwitter}}


\begin{document}
\begin{frame}[fragile]{}
    \titlepage
\end{frame}

\vspace{3em}

\begin{frame}[t]{}
    \center
    \vspace{3em}
    \tableofcontents[
        subsectionstyle=hide/hide/hide,
        sectionstyle=show
        ]
\end{frame}

\section{Python Interfaces with Parallel Computing}
    \subsection{Python? }
    \begin{frame}[t]{Python? What for?}
        \setbeamercolor{block body}{bg=gray!80}
        \setbeamercolor{frametitle}{bg=gray!80}
        \setbeamercolor{block title}{bg=gray!80}
        \begin{figure}[htpb]
            \vspace*{-0.5cm}
            \hspace*{-1.0cm}
            \includegraphics[width=1\linewidth]{media/psf_usage_survey.png}
            \vspace*{-0.5cm}
            \label{fig:media/better_screenshot_python_development_results}
        \end{figure}
        \centering{Python usage among developers
    \footnote{\tiny source: https://www.jetbrains.com/research/python-developers-survey-2018/}
        }
    \end{frame}

    \begin{frame}{A growing scientific computing ecosystem}
        \begin{tikzpicture}
        \node[align=left] (inst) at (1, 0) {%
        \includegraphics[scale=0.40]{media/scikit-learn-logo.png}};
        \node[align=left] (inst) at (1, -1) {%
        \includegraphics[scale=0.4]{media/used_by_scikit-learn.png}};
        \node[align=left] (inst) at (8.5, 0) {%
        \includegraphics[scale=0.30]{media/numpy-logo.png}};
        \node[align=left] (inst) at (8.5, -1.5) {%
        \includegraphics[scale=0.4]{media/used_by_numpy.png}};
        \node[align=left] (inst) at (4.5, -3) {%
        \includegraphics[scale=0.40]{media/pandas.png}};
        \node[align=left] (inst) at (4.5, -4.3) {%
        \includegraphics[scale=0.4]{media/used_by_pandas.png}};
        \end{tikzpicture}
    \end{frame}

    \begin{frame}[t]{Parallel computing? Why?}
        \small
        \vspace{1cm}
        independent, similar computation happens a lot in machine learning. We
        call it embarassingly parallel tasks.
        \vspace{1cm}
        \begin{itemize}
            \item cross validation
            \item multi-class classification
            \item random forests \tiny{(in general: ensemble models)}
        \end{itemize}
        \vspace{1cm}
        \begin{flushright}
            \small
        Exists for many scikit-learn estimators, but not all.
        \end{flushright}
    \end{frame}

    \begin{frame}[fragile]{User API}
        \small
        Parallelization can be enabled using the \verb|n_jobs| option.
        \begin{beamerboxesrounded}{}
            \begin{minted}[bgcolor=beige, fontsize=\tiny]{python}
clf = LogisticRegression(solver='saga', n_jobs=4)
X, y = get_data()
clf.fit(X, y)  # runs on 4 cores!
            \end{minted}
        \end{beamerboxesrounded}{}
        \textit{" Let us know if you want to use more CPU resources, because
        parallelizing your models is not hard for you"}
    \end{frame}

    \begin{frame}[fragile]{Multithreading or multiprocessing?}
        Should these separate computations be ran
        \begin{itemize}
            \item using multiple threads in the main process?
            \item using multiple processes?
        \end{itemize}
    \end{frame}

    \begin{frame}[t]{Cross-validation}
        \footnotesize
        I want to fit the same machine learning model on different subsets of
        the same data.
        \vspace{1cm}
        \begin{columns}
            \begin{column}{0.5\textwidth}
                \only<2-5>{%
                    \small \textbf{thread-based parallelism:} \\
                \footnotesize
                \vspace{0.5cm}
                +: threads share memory
                \begin{itemize}
                    \item no data copies
                    \item no data transfer
                \end{itemize}
                \vspace{0.2cm}
                -: Python forces threads to run sequentially
                }
                \only<6->{%
                \scriptsize
                    \small \textbf{process-based parallelism:} \\
                    \footnotesize
                \vspace{0.5cm}
                +: processes are assured to run in parallel
                \vspace{0.2cm} \\
                -: need to pass and copy data around
                \begin{itemize}
                    \item larger memory footpring
                    \item data transfer overhead
                \end{itemize}
                }
                \vspace{2.3cm}
            \end{column}
            \begin{column}{0.6\textwidth}
                \begin{tikzpicture}[scale=0.55]
                    \tikzset{rounded corners=1}
                \node[align=left, visible on=<2->] (main-thread) at (5, 5.7) {%
                        \only<2-5>{\mycode{main thread}}
                        \only<6->{\mycode{main process}}};
                \node[align=left, visible on=<2>] (all) at (5, 4) {%
                    \resizebox{0.13\textwidth}{!}{%
                    \begin{tabular}{c c}
                        a & b \\
                        \hline
                        \hline
                        1 & 2 \\
                        3 & 4 \\
                        5 & 6 \\
                        7 & 8 \\
                \end{tabular}}};
                \node[align=left, visible on=<3>] (all) at (5, 4) {%
                    \resizebox{0.13\textwidth}{!}{%
                    \begin{tabular}{c c}
                        a & b \\
                        \hline
                        \hline
                        \rowcolor{beige}
                        1 & 2 \\
                        3 & 4 \\
                        \rowcolor{beige}
                        5 & 6 \\
                        \rowcolor{beige}
                        7 & 8 \\
                \end{tabular}}};
                \node[align=left, visible on=<4>] (all) at (5, 4) {%
                    \resizebox{0.13\textwidth}{!}{%
                    \begin{tabular}{c c}
                        a & b \\
                        \hline
                        \hline
                        1 & 2 \\
                        \rowcolor{aliceblue}
                        3 & 4 \\
                        \rowcolor{aliceblue}
                        5 & 6 \\
                        \rowcolor{aliceblue}
                        7 & 8 \\
                \end{tabular}}};
                \node[align=left, visible on=<5>] (all) at (5, 4) {%
                    \resizebox{0.13\textwidth}{!}{%
                    \begin{tabular}{c c}
                        a & b \\
                        \hline
                        \hline
                        \rowcolor{chartreuse}
                        1 & 2 \\
                        \rowcolor{chartreuse}
                        3 & 4 \\
                        \rowcolor{chartreuse}
                        5 & 6 \\
                        7 & 8 \\
                \end{tabular}}};
                \node[align=left, visible on=<6->] (all) at (5, 4) {%
                    \resizebox{0.13\textwidth}{!}{%
                    \begin{tabular}{c c}
                        a & b \\
                        \hline
                        \hline
                        1 & 2 \\
                        3 & 4 \\
                        5 & 6 \\
                        7 & 8 \\
                    \end{tabular}}};
                \node[align=left, visible on=<7->] (cv1) at (1.5, 0) {%
                    \resizebox{0.13\textwidth}{!}{%
                        \begin{tabular}{c c}
                            a & b \\
                            \hline
                            \hline
                            \rowcolor{beige}
                            1 & 2 \\
                              &   \\
                            \rowcolor{beige}
                            5 & 6 \\
                            \rowcolor{beige}
                            7 & 8 \\
                    \end{tabular}}};
                \node[align=left, visible on = <7->] (cv2) at (5, 0) {%
                    \resizebox{0.13\textwidth}{!}{%
                    \begin{tabular}{c c}
                        a & b \\
                        \hline
                        \hline
                          &   \\
                        \rowcolor{aliceblue}
                        3 & 4 \\
                        \rowcolor{aliceblue}
                        5 & 6 \\
                        \rowcolor{aliceblue}
                        7 & 8 \\
                \end{tabular}}};
                \node[align=left, visible on=<7->] (cv3) at (8.5, 0) {%
                    \resizebox{0.13\textwidth}{!}{%
                        \begin{tabular}{c c}
                            a & b \\
                            \hline
                            \hline
                            \rowcolor{chartreuse}
                            1 & 2 \\
                            \rowcolor{chartreuse}
                            5 & 6 \\
                            \rowcolor{chartreuse}
                            3 & 4 \\
                              &   \\
                    \end{tabular}}};
                \node[align=left, visible on = <2->] (worker3) at (8.5, 1.6){%
                % \only<-4>{\mycode{Thread-3}}};
                \temporal<6-7>{\mycode{Thread-3}}{\mycode{Process-3}}{\mycode{Machine-3}}};

                \node[align=left, visible on = <2-8>] (worker2) at (5, 1.6){%
                % \only<-4>{\mycode{Thread-2}}};
                \temporal<6-7>{\mycode{Thread-2}}{\mycode{Process-2}}{\mycode{Machine-2}}};

                \node[visible on = <2->] (worker1) at (1.5, 1.6){%
                % \only<-4>{\mycode{Thread-1   }}};
                    % \only<-4>{\mycode{Thread-1   }}
                    % \only<-4>{\mycode{Thread-1   }}
                \temporal<6-7>{\mycode{Thread-1}}{\mycode{Process-1}}{\mycode{Machine-1}}};
                % {\draw[->, visible on=<5->] (all.south)--(cv1.north east);}
                % {\draw[->, visible on=<6->] (all.south)--(cv2.north);}
                % {\draw[->, visible on=<7->] (all.south)--(cv3.north west);}
                {\draw[->, visible on=<3>] (worker1.east)--(all.south);}
                {\draw[->, visible on=<4>] (worker2.north)--(all.south);}
                {\draw[->, visible on=<5>] (worker3.west)--(all.south);}
                % {\draw[->, visible on=<3->] (all.south)--(cv2.north);}
                % {\draw[->, visible on=<4->] (all.south)--(cv3.north west);}
                \node[visible on =<2-5>, draw,inner sep=5mm,fit=(all) (worker2) (worker3) (worker1)] {};
                \node[visible on =<6->, draw,inner sep=0.5mm,fit=(main-thread) (all) (all) (all)] {};
                \node[visible on =<7->, draw,inner sep=0.5mm,fit=(worker1) (cv1) (cv1) (cv1)] {};
                \node[visible on =<7->, draw,inner sep=0.5mm,fit=(worker2) (cv2) (cv2) (cv2)] {};
                \node[visible on =<7->, draw,inner sep=0.5mm,fit=(worker3) (cv3) (cv3) (cv3)] {};


                \node[visible on = <6->, align=left] (legend) at (4, -3){};
                \node[visible on= <6->, draw,inner sep=1mm,fit=(legend) (legend) (legend) (legend)] {};
                \node[visible on = <6->, align=left] (title) at (7, -3){\tiny : single python interpreter};

                \end{tikzpicture}
            \end{column}
        \end{columns}
    \end{frame}

    % \begin{frame}[t]{}
    %     \center
    %     \vspace{3em}
    %     \tableofcontents[
    %         % currentsubsection,
    %         % hideothersubsections,
    %         subsectionstyle=show/hide/hide,
    %         sectionstyle=show/hide
    %         ]
    % \end{frame}

    \begin{frame}[t]{The challenges of multiprocessing (and beyond)}
        \vspace{1cm}
        Improvements in python multiprocessing mostly concern:
        \hspace{10em}
        \begin{tabular}{m{0.5cm} m{10cm}}
                 & \\
                 & \\
                \includegraphics[width=\linewidth] {media/rocket-emoji.png} & speed (of data communication) \\
                \includegraphics[width=\linewidth]{media/recyclable_emoji.png} & memory footprint (of duplicated data) \\
                \includegraphics[width=\linewidth]{media/green-tick-emoji.png} & ease of use, safety (deadlocks) \\
            \end{tabular}

    \end{frame}

    \begin{frame}[t]{}
        \vspace{5em}
        \center disclaimer: this talk is mostly cpython specific.
    \end{frame}

    \begin{frame}[t]{In practice}
        \begin{figure}[htpb]
            \centering
            \only<1>{\includegraphics[width=\linewidth]{media/plots/benchmark_plots/seq_vs_parallel_bounds_only.png}}%
            \only<2>{\includegraphics[width=\linewidth]{media/plots/benchmark_plots/seq_vs_parallel_bounds_and_scatter.png}}%
            \only<3>{\includegraphics[width=\linewidth]{media/plots/benchmark_plots/seq_vs_parallel_bounds_and_scatter_and_patches.png}}
            % \caption{Seq vs parallel bounds only}
            \label{seq_vs_parallel_bounds_only}
        \end{figure}
    \end{frame}

    % \subsection{At which level?}
    %     \begin{frame}[t]{The Python Interpreter}

    %     \end{frame}
    %     \begin{frame}[t]{Parallelism at the Python level}

    %     \end{frame}
    %     \begin{frame}[t]{Parallelism at the C level}
    %     \end{frame}
    % \subsection{Using which code?}
    %     \begin{frame}[t]{The python virtual machine}

    %     \end{frame}
    %     \begin{frame}[t]{Arbirtray code execution}

    %     \end{frame}
    %     \begin{frame}[t]{Consequences for paralle computing}

    %     \end{frame}
    % \subsection{In practice?}
    %     \begin{frame}[t]{several matrix-matrix multiplication}
    %         \begin{itemize}
    %             \item using python threads
    %             \item using python processes
    %             \item using BLAS parallelism
    %         \end{itemize}
    %     \end{frame}



\section{Built-in and Third Party ressources}
    \subsection{In the Standard Library}

    \begin{frame}[fragile]{\mycode[\small]{multiprocessing}}
        provides all necessary constructs for creating and handling proccesses
        in \mintinline[bgcolor=gray!10]{python}{Python} programs
        \vspace{1em}
        \begin{columns}
            \begin{column}{0.5\textwidth}
                \begin{itemize}
                    \visible<2->{\item creation}
                    \visible<3->{\item communication}
                    \visible<4->{\item \hl{shared memory}}
                \end{itemize}
                \vspace{2cm}
            \end{column}
            \begin{column}{0.5\textwidth}
                \begin{tikzpicture}
                    \tikzset{rounded corners=1}
                \draw (0, 0) node[rectangle, visible on=<2->, draw=gray, font=\tiny] (main){main process};
                \draw (0, -4) node[rectangle, visible on=<4->, draw=gray, font=\tiny,alt=<4>{fill=yellow}{fill=gray!10} ]
                    (lock){\mintinline[fontsize=\tiny]{python}{mp.Lock}};
                \draw (1.5, -3) node[rectangle, visible on = <2->, draw=gray, font=\tiny,alt=<2>{fill=yellow}{fill=gray!10} ]
                    (child2){\mintinline[fontsize=\tiny]{python}{mp.Process}};
                \draw (-1.5, -3) node[rectangle, draw=gray,visible on = <2->, font=\tiny,alt=<2>{fill=yellow}{fill=gray!10} ]
                    (child1){\mintinline[fontsize=\tiny]{python}{mp.Process}};
                % \draw (0, -2) node[rectangle, inner sep=1pt, draw=gray, font=\tiny] (queue){child process};
                \draw (0, -1.5) node[rectangle split, visible on=<3->, rectangle split parts=3,
                       draw, minimum width=1cm,font=\small,
                      rectangle split part align={center}, inner sep=1pt]
                      (queue) {};
                  \draw (2, -1.5) node[visible on=<3->, alt=<3>{fill=yellow}{fill=gray!10}]
                      (queue label) {\mintinline[fontsize=\tiny]{python}{mp.Queue}};
                \draw[visible on=<3->, ->] (queue.south) -- (child1.north);
                \draw[visible on=<3->, ->] (queue.south) -- (child2.north);
                \draw[visible on=<3->, ->] (main.south) -- (queue.north);
                \draw[visible on=<4->, ->] (child2.south) -- (lock.north);
                \draw[visible on=<4->, ->] (child1.south) -- (lock.north);
                % \node[visible on = <7>, align=left] (backend) at (2, 5){\footnotesize using many machines};
                \end{tikzpicture}
            \end{column}
        \end{columns}
        \visible<4>{It's a very rich library!}
    \end{frame}

    \begin{frame}[fragile]{\mycode[\small]{multiprocessing}}
        Programs executing embarassingly parallel tasks share a common
        multiprocessing strucure:
        \begin{columns}
            \begin{column}{0.55\textwidth}
                \uncover<1->{
                \inputminted[fontsize=\tiny,bgcolor=beige]{python}{scripts/multiprocessing_q.py}
                }
                \uncover<2->{
                \vspace{-3.1em}
                \inputminted[fontsize=\tiny,bgcolor=beige]{python}{scripts/multiprocessing_qa.py} }
            \end{column}
            \begin{column}{0.5\textwidth}
                \hspace{1em}
                \visible<2->{
                \begin{tikzpicture}
                    % \tikzset{rounded corners=1}
                \draw (0, 0) node[rectangle, rounded corners, draw=gray, font=\tiny, minimum width=80pt] (main){main process};
                \draw (0, -3) node[rectangle, rounded corners, draw=gray, font=\tiny] (child1){%
                    \mintinline[fontsize=\tiny]{python}{worker-1}};
                \draw (0, -4) node[rectangle, rounded corners, draw=gray, font=\tiny] (child2){%
                    \mintinline[fontsize=\tiny]{python}{worker-2}};
                % \draw (0, -2) node[rectangle, inner sep=1pt, draw=gray, font=\tiny] (queue){child process};
                \draw (-1, -1.5) node[rectangle split, rectangle split parts=2,
                       draw, minimum width=1cm,font=\small,
                      rectangle split part align={center}, inner sep=1pt]
                      (queue) {%
                          \mintinline[fontsize=\tiny]{python}{greet("Alice")}
                          \nodepart{two} \mintinline[fontsize=\tiny]{python}{greet("Bob")}};

                \draw (1, -1.5) node[rectangle split,
                    rectangle split parts=2,
                       draw, minimum width=1cm,font=\small,
                      rectangle split part align={center}, inner sep=1pt]
                      (result queue) {%
                          \mintinline[fontsize=\tiny]{python}{result_1}
                          \nodepart{two} \mintinline[fontsize=\tiny]{python}{result_2}};
                % \draw[visible on=<2->, ->] (queue.south) edge (-1, -3) (child1);
                      \draw[->] (queue.south) |- ++(0, -1.21) -- (child1.west);
                      \draw[->] (queue.south) |- ++(0, -2.22) -- (child2.west);

                      \draw[->]  (child2.east) |- ++(0.35, 0) -- (result queue.south);
                      \draw[->]  (child1.east)|- ++(0.35, 0) -- (result queue.south);

                      \draw[->, draw=gray]  (result queue) |- ++(0, 1.2);
                      \draw[->, draw=gray]  (main.193) -- ++(0, -1);

                    \node[draw=gray, visible on=<3->, ultra thick, inner sep=3mm,fit=(queue) (child2) (result queue) (queue)] (pool) {};
                    \draw (1, -6) node[font=\tiny, visible on = <3->, fill=gray!10] at (pool.250) {%
                      \mintinline[fontsize=\tiny]{python}{mp.Pool}}; ;
                \end{tikzpicture}
            }
            \end{column}
        \end{columns}
        \uncover<3>{This structure is abstracted away in the \mycode[\small]{mp.Pool}
        class}
    \end{frame}

    \begin{frame}[fragile]{\mycode[\small]{multiprocessing} portability}
        \small
        \mycode[\scriptsize]{multiprocessing} has some issues with:
        \uncover<2->{
        \begin{tabular}{m{0.5cm} m{10cm}}
            & \\
            \includegraphics[width=\linewidth] {media/cross-emoji.png} &
            interactive sessions (\faWindows \, + \mycode[\tiny]{IPython, Jupyter}) \\
        \end{tabular}
        \vspace{-0.5cm}
        \inputminted[fontsize=\tiny,bgcolor=beige]{pycon}{scripts/interactive_session_windows.txt}
        \vspace{-1cm}
        \inputminted[fontsize=\tiny,bgcolor=beige]{pycon}{scripts/interactive_session_windows_tb.txt}

        \vspace{-0.5cm}
        }
        \uncover<3->{%
        \begin{tabular}{m{0.5cm} m{10cm}}
            \includegraphics[width=\linewidth] {media/cross-emoji.png} &
            external libraries (\faApple \, + \mycode{numpy}, \mycode{OpenMP}) \\
        \end{tabular}}
        \uncover<4->{%
        \begin{tabular}{m{0.5cm} m{10cm}}
            \includegraphics[width=\linewidth] {media/cross-emoji.png} & recovering from child processes crashes
        \end{tabular}}
    \end{frame}

    \begin{frame}[t]{\mycode[\small]{loky}}
        \vspace{1em}
        \mycode[\small]{loky} is a third party package, that provides a more
        robust process pool implementation. It works out of the box on:
        \begin{tabular}{m{0.5cm} m{10cm}}
            & \\
            \includegraphics[width=\linewidth] {media/green-tick-emoji.png} &
            All \mycode[\scriptsize]{Python} versions \\ \includegraphics[width=\linewidth] {media/green-tick-emoji.png} &
            All Operating Systems \\
            \includegraphics[width=\linewidth] {media/green-tick-emoji.png} &
            Interactive shells\\
            & \\
        \end{tabular}
        \vspace{1em}
        It is also the official backend of \mycode{scikit-learn}

    \end{frame}

    \begin{frame}[fragile]{wait...}
        \mycode{loky} is built on top of another standard library module:
        \mycode{concurrent.futures}. The features of loky are regularly ported
        to \mycode{concurrent.futures} upstream.

        \begin{flushright}
            Why not building on top of \mycode{mp.pool}?
        \end{flushright}
        mostly because \mycode{mp.Pool} is suspected to have unfixeable deadlocks.

    \end{frame}

    \begin{frame}[fragile]{User API}
        \scriptsize \mycode{concurrent futures} and \mycode{loky} \textbf{only expose (the same) worker pool} objects
        \bigskip


        \uncover<2->{
            \alt<3>{
        using \mycode{loky}
        \smallskip
        \inputminted[fontsize=\tiny,bgcolor=beige]{python}{scripts/loky.py}
        }{
        using \mycode{concurent.futures}
        \smallskip
        \inputminted[fontsize=\tiny,bgcolor=beige]{python}{scripts/concurrent_futures.py}
        }}
    \end{frame}

    \begin{frame}[t]{joblib}
    \end{frame}
    \subsection{From one to another}
        \begin{frame}[t]{going upstream}

        \end{frame}

\section{Recent Advances}
\subsection{Why we need better IPC}
\subsection{Better extensibility of the pickle module}
\subsection{PEP 574}
    \begin{frame}[t]{principle}
    \end{frame}

    \begin{frame}[t]{better memory footrpint}
    \end{frame}

    \begin{frame}[t]{better speed}
    \end{frame}
\subsection{built-in shared memory in the standard library}



\begin{frame}[fragile]{}
    \setbeamercolor{block body}{bg=beige}
    \begin{beamerboxesrounded}{}
        \inputminted[bgcolor=beige]{python}{scripts/test_script.py}
    \end{beamerboxesrounded}
\end{frame}
\end{document}
